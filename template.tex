\documentclass[paper=a4, fontsize=11pt, twocolumn]{scrartcl}

%%% LANGUAGE SETTINGS %%%
\usepackage[ngerman]{babel}							% German language/hyphenation
\usepackage[german]{babelbib}

\usepackage[utf8]{inputenc}
\usepackage[left=3cm,right=2cm,top=2cm,bottom=2cm,includeheadfoot]{geometry}
\usepackage[protrusion=true,expansion=true]{microtype}				% Better typography
\usepackage{amsmath,amsfonts,amsthm}							% Math packages

\usepackage[hang, small,labelfont=bf,up,textfont=it,up]{caption}			% Custom captions under/above floats

\usepackage[pdftex]{graphicx}										% Enable pdflatex

\usepackage{booktabs}											% Nicer tables
\usepackage{tabularx}

\usepackage[colorlinks=false,breaklinks=true,backref=none]{hyperref}
\usepackage{url}

\usepackage{listings}

\usepackage{abstract}
\usepackage{blindtext}

\bibliographystyle{acm}

%%% Advanced verbatim environment
\usepackage{verbatim}
\usepackage{fancyvrb}
\DefineShortVerb{\|}											% delimiter to display inline verbatim text

%%% Custom sectioning (sectsty package)
\usepackage{sectsty}											% Custom sectioning (see below)
\allsectionsfont{%												% Change font of al section commands
	\usefont{OT1}{bch}{b}{n}%									% bch-b-n: CharterBT-Bold font
}

\sectionfont{													% Change font of \section command
	\usefont{OT1}{bch}{b}{n}										% bch-b-n: CharterBT-Bold font
	\sectionrule{0pt}{0pt}{-5pt}{0.8pt}								% Horizontal rule below section
}

%%% Custom headers/footers (fancyhdr package)
\usepackage{fancyhdr}
\pagestyle{fancyplain}
\fancyhead{}													% No page header
\fancyfoot[C]{\thepage}											% Pagenumbering at center of footer
\renewcommand{\headrulewidth}{0pt}								% Remove header underlines
\renewcommand{\footrulewidth}{0pt}								% Remove footer underlines
\setlength{\headheight}{13.6pt}

%%% Equation and float numbering
\numberwithin{equation}{section}									% Equationnumbering: section.eq#
\numberwithin{figure}{section}										% Figurenumbering: section.fig#
\numberwithin{table}{section}										% Tablenumbering: section.tab#



%%% Title
\title{ \vspace{-1in} 	\usefont{OT1}{bch}{b}{n}
		\huge \strut  Seminar: Peer-To-Peer Overlay Network Systems S15\strut \\
		\Large \bfseries \strut Freenet: A Distribyted Anonymous Information Storage and Retrieval System \strut
}

\author{ 					\usefont{OT1}{bch}{m}{n}
  Dennis Hägler\\		\usefont{OT1}{bch}{m}{n}
  Freie Universität Berlin\\	\usefont{OT1}{bch}{m}{n}
  \texttt{dennis.haegler@fu-berlin.de}
}

%%% Begin document
\begin{document}
\twocolumn[
\begin{@twocolumnfalse}
\maketitle
\begin{abstract}
  - was sagt das paper
  - was ist sicherheitsprinzipien
  - was sind Analyseprinzipien
  - bestehen aus Protokollen, die aus atomic primitives bestehen
  - wenn atomic save, und richtig genutzt dann auch protocol save
  - was ist mit atomic gemeint
\vspace{4em}
\end{abstract}
\end{@twocolumnfalse}
]

\section{Einleitung}

% \section{Architektur}
Ein Freenet-Netzwerk besteht aus Nutzer-Knoten, die ihren lokalen Speicher
anderen Nutzern zur Verfügung stellen. Jeder Knoten hat einen direkten
Nachbarn, von dem gelesen wird und einen direkten Nachbarn, zu dem Dateien
verteilt werden. Des Weiteren verwaltet jeder Knoten eine dynamische
Routingtabelle.  Jede Routingtabelle beinhaltet die Adressen von Knoten und
deren Schlüssel.  Genutzt wird Freenet, um die eigene Speicherkapazität zu
erhöhen und um ungenutztem Speicher zur Verfügung zu stellen.

Anfragen für Schlüssel von Knoten zu Knoten erfolgen über eine Kette von
Proxy-Requests, indem jeder Knoten entscheiden kann, an wen der nächste Request
geht.  Dieses Verfahren ähnelt dem IP-Routing. Jede Route ändert sich abhängig
vom angefragten Schlüssel.  Um die Privatsphäre der Knoten zu unterstützen,
haben diese nur Informationen über den ummittelbaren Nachbarn (up- und
downstream).

Jeder Request erhält einen hops-to-live Wert, der sich bei jedem erreichten
Knoten verringert. Dadurch werden Endlosschleifen vermieden.  Jeder Request
enthält außerdem eine pseudo-unique Zufalls-ID, um Endlossschleifen zu
verhindern. Jeder Request, der zum zweiten mal bei einem Knoten ankommt, fliegt
raus. Es gibt keine Herarchie der Knoten. Alle Knoten sind gleichgestellt im
Freenet.

\subsection{Schlüssel} Dateien im Freenet werden mit einem 160 Bit SHA-1
Schlüssel dargestellt.  Freenet hat 3 verschiedene Schlüssel, die einen
bestimmten Teil einer Datei beschreiben.

\subsubsection{Keyword-Signed Schlüssel} Der Keyword-Signed Schlüssel
dient für die Beschreibung einer Datei im Freenet. Jeder Benutzer, der eine
Datei anlegt, wählt einen Text zur Idenfizierung dieser Datei, wie z.B.
\begin{lstlisting}
text/philosophy/sun-trz/art-of-war
\end{lstlisting}
Aus diesem Text wird deterministisch ein public-private Schlüsselpaar
generiert.  Der Public Schlüssel wird anschließend gehasht, um den
Dateischlüssel zu erhalten, der dann mit dem privaten Schlüssel signiert wird.
Der Dateischlüssel ist der Schlüssel, mit der die Datei dargestellt wird.
Dieser Schlüssel kann mit dem gewählten Text erneut berechnet und gesucht
werden. Um anderen Nutzern den Zugriff auf die Datei zu gewähren, verschickt
der Autor der Datei den beschreibenden String. Keyword-Signed Schlüssels sind
so leicht zu merken. Ein Problem bei dem Keyword-Signed Schlüssel ist der
geringe Namespace, welcher dazu frühren kann, das unterschiedliche Dateien die
selbe Bezeichnung und dadurch den selben Dateischlüssel erhalten. Um das
Problem mit der Namenskollidierung zu umgehen dient der Signed-Subspace
Schlüssel.

\subsubsection{Signed-Subspace Schlüssel}
Der Signed-Subspace Schlüssel ist eine Erweiterung des Keyword-Signed
Schlüssel. Der Schlüsselt erweitert den Namespace und reduziert Kollisionen, wo
zwei verschieden Dateien den selben Schlüssel erhalten können.  Ein zufällig
gewähltes asymmetrisches Schlüsselpaar stellt den Namespace dar.  Um eine Datei
hinzuzufügen, wählt der Nutzer einen beschreibenden Text. Der public
Signed-Subspace Schlüssel und der beschreibende Text werden unabhängig
voneinander gehasht, anschließend miteinander geXOR't und dann wieder gehasht.

\subsubsection{Content-Hash Schlüssel}
Der Content-Hash Schlüssel repräsentiert den Inhalt der Datei in Form eines
Hashes. Jede Datei erhält dadurch einen pseudo-einzigartigen Schlüssel.
Zusätzlich wird eine Datei mit einem zufällig generierten Schlüssel
verschlüsselt.  Um Zugriff auf die Datei zu ermöglichen, veröffentlicht der
Nutzer den Content-Hash Schlüssel und den dazu gehörigen Schlüssel zur
Entschlüsselung.

Eine Datei kann kann mit dem Content-Hash Schlüssel durch den Autor
aktualisiert werden. Dazu wird ein neuer Content-Hash Schlüssel erzeugt und
unter dem Signed-Subspace Schlüssel abgelegt. Der neue Schlüssel muss sich vom
alten unterscheiden. Sollte die neue Datei einen Knoten erreichen, der noch auf
die alte Version verweist, ensteht eine Kollision, die dazu führt, dass der
Knoten die Signatur der neuen Version überprüft, mit der alten Version
validiert und auf die neue Version verweist. Der Signed-Subspace Schlüssel
verweist immer auf die aktuellste Version der Datei. Die alte Version kann
weiterhin erreicht werden, indem direkt der Content-Hash Schlüssel der älteren
Version angefragt wird. Wird die ältere Version über einen bestimmten Zeitraum
nicht mehr aufgerufen, kann diese vom Freenet entfernt werden.

Eine weitere Funktion vom Content-Hash Schlüssel ist das Aufteilen einer Datei
in mehrere Teildateien, um z.B. große Daten mit weniger Bandbreite über das
Netzwerk zu verschicken.

% \subsection{Datenaustausch}
\subsubsection{Daten empfangen}
Um Daten im Freenet zu empfangen, wählt oder berechnet der Nutzer einen Hash.
Dieser Hash wird dann als Request-Nachricht über eine Kette von Requests und
mit einer gegebenen hops-to-live-Zahl versendet. Als Start dient der eigene Knoten.
Der Knoten, bei dem die Request-Nachricht ankommt, überprüft, ob der gesuchte
Hash vorhanden ist. Sollte die Datei gefunden werden, sendet der Knoten
die Datei zurück. Sollte der gesuchte Schlüssel nicht vorhanden sein, wird
der nächste Knoten aus der Routingtabelle genommen, um die Request-Nachricht
weiterzuleiten. Dabei wird die hops-to-live-Zahl um eins verringert. Dieser
Vorgang wiederholt sich so lange, bis der hops-to-live-Wert auf null fällt.
Kann ein Knoten einen Request nicht mehr weiterleiten, wird der vorherige
Knoten benachrichtigt. Dieser sucht sich einen neuen Knoten aus der
Routingtabelle und versendet den Request neu. Sollte der hops-to-live-Wert auf
null fallen und es wurde keine Datei mit dem Schlüssel gefunden, wird eine
Fehlernachricht bis zum Anfang der Kette weitergeleitet.
Wird die Datei gefunden, wird dieser Hash in allen Knoten eingetragen, die
Bestandteil der Request-Kette waren.

\subsubsection{Daten verteilen}
Verteilen von Daten ist ähnlich wie das Empfangen von Daten. Der Nutzer
berechnet den Hashwert für die Datei und versendet diese an den eigenen Knoten.
Dieser überpüft, ob der Hashwert schon vorhanden ist. Sollte das der Fall sein,
kommt es zur Kollision und der Nutzer muss einen neuen Schlüssel berechnen.
Sollte der Schlüssel nicht vorhanden sein, leitet der Knoten den Request
weiter, bis der hop-to-live-Wert aufgebraucht ist oder die Datei gefunden
wurde. Sollte die Datei gefunden werden, kommt es wieder zur Kollision und die
Fehlernachricht wird bis zum Initialknoten durchgereicht, der dann wieder einen
neuen Schlüssel berechnen muss. Sollte die Datei nicht gefunden werden, wird
die Datei in allen Knoten eingefügt, die an der Request-Kette beteiligt waren.
Sollte ein Knoten keinen Speicherplatz für die neue Datei verfügbar haben,
wird die am wenigsten aufgerufene Datei ersetzt. Freenet garantiert somit keine
permanente Persistierung von Daten sicher.

\subsubsection{Daten verwalten}
Der Speicher eines Knotens wird als LRU (Last Recent Used) cache verwaltet.
Die Dateien sind nach letzt genutzt sortiert. Kommt beim Knoten eine neue Datei
zum speichern und der Speicher ist voll, wird die am wenigsten genutzte Datei
vertrieben, bis wieder Speicherplatz vorhanden ist. Der Eintrag in der
Routingtablle bleibt vorerst bestehen. Somit kann eine Kopie der Datei
wieder beschafft werden, wenn genug Speicherplatz vorhanden ist. Sollte sich
der Knoten mit weiteren neuen Dateien füllen, verschwindet auch der Eintrag
in der Routingtabelle der entfernten Datei.

Dateien im Freenet sind nicht permanent verfügbar. Sollten alle Knoten
entscheiden die selbe Datei aus ihren Speicher zu vertreiben, ist die im
kompletten Netzwerk nicht mehr verfügbar.

Wird eine veraltete Datei weiterhin genutzt, bleibt sie solange im Speicher,
wie sie auch abgefragt wird. Jede Datei ist im Freenet verschlüsselt und
benötigt einen Schlüssel zum Entschlüsseln. Dieser Schlüssel ist völlig
unabhängig von dem Content-Hash-Key.
Ohne Schlüssel kennt der Betreiber den Inhalt der Datei nicht und ist somit
rechtlich abgeschützt, da jedes Wissen vom Inhalt abgestritten werden kann.

\subsection{Knoten hinzufügen}
Ein neuer Knoten wählt zufälligen Seed und verschickt diesen als Nachricht über
das Freenet-Netzwerk. In dieser Ankündigungsnachricht ist die Adresse und dem
Hash des Seeds des neuen Knoten enthalten.

Bei Einkunft einer Ankündigung eines neuen Knotens, generiert der Empfänger
einen zufälligen Seed, XOR'ed diesen mit dem empfangenem Hash und hasht das
Ergebnis um eine Verbindlichkeit herzustellen. Danach leitet der Empfänger
Knoten den neuen Hash zu einem zufällig gewähltem Knoten aus seiner eigenen
Routingtabelle. Dieser Prozess wiederholt sich bis der hops-to-live Wert von
der Ankündigung des neuen Knoten aufgebraucht ist.
Der letzte Knoten, der die Ankündigungsnachricht empfängt, generiert nur einen
Seed.
Nach Überprüfung der Vereinbarungen in den jeweiligen Knoten, fügt jeder Knoten
in der Kette den neuen Knoten in seine Routingtabelle ein.

% \section{Protokoll Details}
Jede Nachricht im Freenet ist Paket-orientiert und beinhaltet selbst erklärende
Nachrichten, sowie eine Transaktions-ID.
Adressen von Knoten beinhalten eine Methode für den Transport und einen
speziellen Identifikator wie zum Beispiel eine IP und eine Port-Nummer.

\subsection{Transaktionen}
Eine Transaktion beginnt mit einer Request-Handshake Nachricht von einem Knoten
zum anderen. Ist der angefragte Knoten aktiv, antwortet dieser mit einer
Reply-Nachricht und spezifiziert die Protokollversion, die der Knoten versteht.
Handshakes bleiben mehrere Stunden bestehen, damit der Handshake zwischen den
beiden selben beiden Knoten zu einem späteren Zeitpunkt ausgelassen werden
kann.

\subsection{Nachrichten}
Alle Nachrichten im Freenet beinhalten einen zufällig generierten 64-Bit
Transaktions-ID, eine hops-to-live Wert und einen Depth-Counter.
Zu beachten ist, dass nicht garantiert ist, dass die Transaktions-ID eindeutig
ist. Die daraus resultierenden Kollision sind bei einer limitiertem Anzahl von
Knoten sehr gering. Der hops-to-live Wert wird immer vom Absender der Nachricht
festgelegt und verringert sich bei jeden angekommen Knoten um eins. Der
Depth-Counter erhöht um eins bei jedem Knoten.

Um Daten zu empfangen, sendet der zu sendende Knoten eine Request-Data
Nachricht mit einer Transaktions-ID, einem hops-to-live Wert, einem
Depth-Counter und Schlüssel der Datei nach der gesucht wird.
Der Empfängerknoten überprüft den Speicher nach der gesuchten Datei. Sollte
die Datei nicht im Speicher des Knoten liegen, wird die Anfrage an einem Knoten
aus der Routingtabelle weitergeleitet.

Ist ein Request nach einer Datei erfolgreich, antwortet der angefragte Knoten mit
einer Send-Data Nachricht. Beinhaltet in dieser ist die angefragte Datei und
die Adresse des Knotens, welches diese Datei auslieferte.
Ist ein Request nach einer Datei erfolglos und der hops-to-live Wert
aufgebraucht, antwortet der entfernte Knoten mit Reply-NotFound.

Um Dateien in Freenet einzufügen, sendet der Senderknoten eine Request-Insert
Nachricht mit einer definierten Transaktions-ID, einem hop-to-live Wert,
einem Depth-Counter und einem vorgeschlagen Schlüssel für die Datei. Der
entfernte Knoten überprüft seinen Speicher nach dem gesuchtem Schlüssel. Sollte
der Schlüssel nicht gefunden werden im Speicher, wird die Nachricht
weitergeleitet.
Führt die Request-Insert Nachricht zu einer Kollision, da es der
vorgeschlagenen Schlüssel schon vergeben ist, sendet der entfernte Knoten
entweder eine Send-Data Nachricht, mit der existierenden Datei als Inhalt, oder
eine Reply-NotFound Nachricht. Die Reply-NotFound Nachricht wird gesendet, wenn
die Datei nicht gefunden wurde, dennoch ein Eintrag in der Routingtabelle auf
die Datei referenziert.
Sollte es bei einer Nachricht zum Einfügen einer Datei zu keiner Kollision
kommen und es sind keine weiteren Knoten vorhanden, bei einen hop-to-live Wert
größer Null, antwortet der entfernte Knoten mit einer Request-Continue
Nachricht. Diese Nachricht soll dazu dienen den Sender mitzuteilen, dass nicht
die Anzahl der Knoten abgefragt werden konnte wie gewünscht.
Sollte das Einfügen ohne Kollision enden, erhählt der Initialknoten eine
Reply-Insert Nachricht, die dem Knoten mitteilt, dass die Datei unter diesem
Schlüssel eingefügt werden kann, dann sendet der Knoten eine Send-Data
Nachricht.

% \section{Sicherheit}
Freenet hat das Ziel die Anonymität der Autoren und Leser von Dateien zu
schützen. Selbst die Dateien müssen im Freenet vor feindlichen Modifikationen
geschützt werden und das Freenet-System muss resistent gegen DOS-Angriffen
sein.

Freenet stützt sich auf drei wesentliche Aspekte, die aus der Ausarbeitung
von Reiter und Rubin stammen \cite{reiterandrubin}. Der erste Aspekt ist der
Punkt der Anonymität. Es soll nicht ermittelt werden können wer eine Nachricht
im Netz abgeschickt hat oder an wem diese Nachricht verschickt wurde.
Der Aspekt ist die Erkennung von potentiellen Feinden im Freenet. Zu den
potenziellen Angreifern gehören, lokale Lauscher, bösartige Knoten oder eine
Zusammenarbeit von bösartigen Knoten. Die dritte Achse ist der Grad der
Anonymität.

\subsection{Schlüssel}
Im Freenet können die angefragten Schlüssel nicht versteckt werden, da
sie ein Bestandteil des Routingmechanismuses sind und jede Routing-Tabelle auf
die Schlüssel angewiesen ist. Dadurch ist eine Anonymität der Schlüssel nicht
gewährleistet. Der Nutzen von Hashes als Schlüssel stellt dennoch eine hohe
Unklarheit gegen gelegentlichen Mithörern auf.
Gegen Wörterbuchangriffen bleibt sind die Schlüssel dennoch anfällig.

\subsection{Anonymität}
Die Anonymität von Sender-Knoten in einem Bund aus böswilligen Knoten wird
gewährleistet, da ein Knoten der eine Nachricht erhält, nicht bestimmen kann
ob der Vorgänger die Nachricht initialisierte oder einfach nur
weiterleitete. Durch das zufällige Setzen des Depth-Counters wird das
Zurückverfolgen von Requests drastisch erschwert.

Es gibt keinen Schutz gegenüber lokalen Mithörern, die Nachrichten vom Nutzer
bis zum ersten Knoten beobachten. Solange der erste Knoten ein mithörender
Knoten sein könnte, wird es empfohlen, dass den ersten Knoten für
den Einstieg ins Freenet von der eigenen Maschine zu wählen. Nachrichten
zwischen Knoten sind verschlüsselt und können nicht mehr abgehört werden. Die
einzige Beobachtung bleibt der Ausgang einer Nachricht den Eingang einer
Anfrage, was dazu schließen lass kann, das der Knoten der Absender sein könnte.

\subsection{Datenquellen}
Datenquellen können geschützt werden indem die Felder der Datenquellen
regelmäßig zurückgesetzt werden.  Im Freenet ist es nicht möglich die Quelle
einer Datei zu ermitteln. Liefert der Downstream-Knoten eine Datei, kann diese
von dem Knoten selbst stammen oder von einem anderen Knoten, die an den
Downstream-Knoten gesendet wurde.

\subsection{Daten Manipulation}
Daten, die unter den Content-Hash oder den Signed-Subspace Schlüsseln
gespeichert werden können nicht manipuliert werden, da unechte Dateien im
Freenet nachgewiesen werden können. Die einzige Möglichkeit den Inhalt einer
Datei zu manipulieren besteht, wenn der Angreifer eine Kollision verursacht und
die Signatur fälschen könnte. Unter den Keyword-Signed Schlüssel gespeicherte
Dateien sind anfällig gegen Wörterbuchangriffe.

\subsection{Denial Of Service}
Weitere Angriffe sind die Denial-Of-Service Attacken. Ein Angreifer könnten
versuchen mit einer hohen Anzahl an Abfalldateien das Freenet-Netzwerk zu
fluten. Eine möglicher Konter ist es den Nutzer lange Berechnung vor der
Einfügung durchzuführen zu lassen. Damit würde das Netzwerk entlastet werden.
Ein weiter Methode ist die Aufteilung des Speicher in zwei Segmente.
Ein Speicher dient für neue Dateien und ein Speicher für Dateien, die eine
gewisse Anzahl an Anfragen aufweisen. Mit dieser Methode würden die neuen Datein
nur neue Dateien ersetzen können. Dadurch würden keine Berechnung auf die
etablierten Dateien ausgeführt werden. Die Fluten wird dadurch gestoppt, dass
wenn ein Knoten eine Anfrage versendet, die vom ersten Knoten schon gestoppt
wird, da der Knoten die Datei beinhalten. Eine Verbreitung über das Netzwerk
ist dadurch verhindert.

Ein Ziel von Angreifern kann es sein existieren Dateien im Freenet zu ersetzen.
Wie bei den DOS-Attacken ist ein Angriff auf die Content-Hash oder
Signed-Subspace Schlüssel nicht möglich.

\bibliography{bibliography}
\end{document}
