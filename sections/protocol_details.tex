\section{Protokoll Details}
Jede Nachricht im Freenet ist Paket-orientiert und beinhaltet selbst erklärende
Nachrichten, sowie eine Transaktions-ID.
Adressen von Knoten beinhalten eine Methode für den Transport und einen
speziellen Identifikator wie zum Beispiel eine IP und eine Port-Nummer.

\subsection{Transaktionen}
Eine Transaktion beginnt mit einer Request-Handshake Nachricht von einem Knoten
zum anderen. Ist der angefragte Knoten aktiv, antwortet dieser mit einer
Reply-Nachricht und spezifiziert die Protokollversion, die der Knoten versteht.
Handshakes bleiben mehrere Stunden bestehen, damit der Handshake zwischen den
beiden selben beiden Knoten zu einem späteren Zeitpunkt ausgelassen werden
kann.

\subsection{Nachrichten}
Alle Nachrichten im Freenet beinhalten einen zufällig generierten 64-Bit
Transaktions-ID, eine hops-to-live Wert und einen Depth-Counter.
Zu beachten ist, dass nicht garantiert ist, dass die Transaktions-ID eindeutig
ist. Die daraus resultierenden Kollision sind bei einer limitiertem Anzahl von
Knoten sehr gering. Der hops-to-live Wert wird immer vom Absender der Nachricht
festgelegt und verringert sich bei jeden angekommen Knoten um eins. Der
Depth-Counter erhöht um eins bei jedem Knoten.

Um Daten zu empfangen, sendet der zu sendende Knoten eine Request-Data
Nachricht mit einer Transaktions-ID, einem hops-to-live Wert, einem
Depth-Counter und Schlüssel der Datei nach der gesucht wird.
Der Empfängerknoten überprüft den Speicher nach der gesuchten Datei. Sollte
die Datei nicht im Speicher des Knoten liegen, wird die Anfrage an einem Knoten
aus der Routingtabelle weitergeleitet.

Ist ein Request nach einer Datei erfolgreich, antwortet der angefragte Knoten mit
einer Send-Data Nachricht. Beinhaltet in dieser ist die angefragte Datei und
die Adresse des Knotens, welches diese Datei auslieferte.
Ist ein Request nach einer Datei erfolglos und der hops-to-live Wert
aufgebraucht, antwortet der entfernte Knoten mit Reply-NotFound.

Um Dateien in Freenet einzufügen, sendet der Senderknoten eine Request-Insert
Nachricht mit einer definierten Transaktions-ID, einem hop-to-live Wert,
einem Depth-Counter und einem vorgeschlagen Schlüssel für die Datei. Der
entfernte Knoten überprüft seinen Speicher nach dem gesuchtem Schlüssel. Sollte
der Schlüssel nicht gefunden werden im Speicher, wird die Nachricht
weitergeleitet.
Führt die Request-Insert Nachricht zu einer Kollision, da es der
vorgeschlagenen Schlüssel schon vergeben ist, sendet der entfernte Knoten
entweder eine Send-Data Nachricht, mit der existierenden Datei als Inhalt, oder
eine Reply-NotFound Nachricht. Die Reply-NotFound Nachricht wird gesendet, wenn
die Datei nicht gefunden wurde, dennoch ein Eintrag in der Routingtabelle auf
die Datei referenziert.
Sollte es bei einer Nachricht zum Einfügen einer Datei zu keiner Kollision
kommen und es sind keine weiteren Knoten vorhanden, bei einen hop-to-live Wert
größer Null, antwortet der entfernte Knoten mit einer Request-Continue
Nachricht. Diese Nachricht soll dazu dienen den Sender mitzuteilen, dass nicht
die Anzahl der Knoten abgefragt werden konnte wie gewünscht.
Sollte das Einfügen ohne Kollision enden, erhählt der Initialknoten eine
Reply-Insert Nachricht, die dem Knoten mitteilt, dass die Datei unter diesem
Schlüssel eingefügt werden kann, dann sendet der Knoten eine Send-Data
Nachricht.
