\section{Architektur}
Ein Freenet-Netzwerk besteht aus Nutzer-Knoten, die ihren lokalen Speicher
anderen Nutzern zur Verfügung stellen. Jeder Knoten hat einen direkten
Nachbarn, von dem gelesen wird und einen direkten Nachbarn, zu dem Dateien
verteilt werden. Des Weiteren verwaltet jeder Knoten eine dynamische
Routingtabelle.  Jede Routingtabelle beinhaltet die Adressen von Knoten und
deren Schlüssel.  Genutzt wird Freenet, um die eigene Speicherkapazität zu
erhöhen und um ungenutztem Speicher zur Verfügung zu stellen.

Anfragen für Schlüssel von Knoten zu Knoten erfolgen über eine Kette von
Proxy-Requests, indem jeder Knoten entscheiden kann, an wen der nächste Request
geht.  Dieses Verfahren ähnelt dem IP-Routing. Jede Route ändert sich abhängig
vom angefragten Schlüssel.  Um die Privatsphäre der Knoten zu unterstützen,
haben diese nur Informationen über den ummittelbaren Nachbarn (up- und
downstream).

Jeder Request erhält einen hops-to-live Wert, der sich bei jedem erreichten
Knoten verringert. Dadurch werden Endlosschleifen vermieden.  Jeder Request
enthält außerdem eine pseudo-unique Zufalls-ID, um Endlossschleifen zu
verhindern. Jeder Request, der zum zweiten mal bei einem Knoten ankommt, fliegt
raus. Es gibt keine Herarchie der Knoten. Alle Knoten sind gleichgestellt im
Freenet.

\subsection{Schlüssel} Dateien im Freenet werden mit einem 160 Bit SHA-1
Schlüssel dargestellt.  Freenet hat 3 verschiedene Schlüssel, die einen
bestimmten Teil einer Datei beschreiben.

\subsubsection{Keyword-Signed Schlüssel} Der Keyword-Signed Schlüssel
dient für die Beschreibung einer Datei im Freenet. Jeder Benutzer, der eine
Datei anlegt, wählt einen Text zur Idenfizierung dieser Datei, wie z.B.
\begin{lstlisting}
text/philosophy/sun-trz/art-of-war
\end{lstlisting}
Aus diesem Text wird deterministisch ein public-private Schlüsselpaar
generiert.  Der Public Schlüssel wird anschließend gehasht, um den
Dateischlüssel zu erhalten, der dann mit dem privaten Schlüssel signiert wird.
Der Dateischlüssel ist der Schlüssel, mit der die Datei dargestellt wird.
Dieser Schlüssel kann mit dem gewählten Text erneut berechnet und gesucht
werden. Um anderen Nutzern den Zugriff auf die Datei zu gewähren, verschickt
der Autor der Datei den beschreibenden String. Keyword-Signed Schlüssels sind
so leicht zu merken. Ein Problem bei dem Keyword-Signed Schlüssel ist der
geringe Namespace, welcher dazu frühren kann, das unterschiedliche Dateien die
selbe Bezeichnung und dadurch den selben Dateischlüssel erhalten. Um das
Problem mit der Namenskollidierung zu umgehen dient der Signed-Subspace
Schlüssel.

\subsubsection{Signed-Subspace Schlüssel}
Der Signed-Subspace Schlüssel ist eine Erweiterung des Keyword-Signed
Schlüssel. Der Schlüsselt erweitert den Namespace und reduziert Kollisionen, wo
zwei verschieden Dateien den selben Schlüssel erhalten können.  Ein zufällig
gewähltes asymmetrisches Schlüsselpaar stellt den Namespace dar.  Um eine Datei
hinzuzufügen, wählt der Nutzer einen beschreibenden Text. Der public
Signed-Subspace Schlüssel und der beschreibende Text werden unabhängig
voneinander gehasht, anschließend miteinander geXOR't und dann wieder gehasht.

\subsubsection{Content-Hash Schlüssel}
Der Content-Hash Schlüssel repräsentiert den Inhalt der Datei in Form eines
Hashes. Jede Datei erhält dadurch einen pseudo-einzigartigen Schlüssel.
Zusätzlich wird eine Datei mit einem zufällig generierten Schlüssel
verschlüsselt.  Um Zugriff auf die Datei zu ermöglichen, veröffentlicht der
Nutzer den Content-Hash Schlüssel und den dazu gehörigen Schlüssel zur
Entschlüsselung.

Eine Datei kann kann mit dem Content-Hash Schlüssel durch den Autor
aktualisiert werden. Dazu wird ein neuer Content-Hash Schlüssel erzeugt und
unter dem Signed-Subspace Schlüssel abgelegt. Der neue Schlüssel muss sich vom
alten unterscheiden. Sollte die neue Datei einen Knoten erreichen, der noch auf
die alte Version verweist, ensteht eine Kollision, die dazu führt, dass der
Knoten die Signatur der neuen Version überprüft, mit der alten Version
validiert und auf die neue Version verweist. Der Signed-Subspace Schlüssel
verweist immer auf die aktuellste Version der Datei. Die alte Version kann
weiterhin erreicht werden, indem direkt der Content-Hash Schlüssel der älteren
Version angefragt wird. Wird die ältere Version über einen bestimmten Zeitraum
nicht mehr aufgerufen, kann diese vom Freenet entfernt werden.

Eine weitere Funktion vom Content-Hash Schlüssel ist das Aufteilen einer Datei
in mehrere Teildateien, um z.B. große Daten mit weniger Bandbreite über das
Netzwerk zu verschicken.
