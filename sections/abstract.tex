Daten werden heutzutage mehr als je zuvor digital abgespeichert. Die
Kommunikation mit unseren Mitmenschen erfolgt zum Großteil über das Internet.
Sei es der Messanger auf dem Smartphones oder das Versenden einer E-Mail.  Wie
kann sichergestellt werden, dass diese Nachrichten nicht von dritten Personen
eingesehen werden? Wie wird verhindert, dass Dokumente, die personenbezogene
Daten enthalten nicht von weiteren Personen gelesen werden?

Um Angriffe auf die Dokumente zu verhindern, werden Daten und Nachrichten
verschlüsselt. Es bestehen Protokolle, die dem Anwender helfen die genannten
Ziele zu erreichen.  Im ersten Kapitel \cite{modern} werden neben den
grundlegende Begriffen wie symmetrische sowie asymmetrische
Verschlüsselungsverfahren, trust model und Protokolle, auch die die Ziele der
Kryptografie sowie die Zusammensetzung eines sicheren Protokolls beschrieben.
Ein weiteres Thema ist die Einstufung eines Protokolls als "sicher".
Protokolle sind eine Ansammlung von Programmen. Diese werden als atomic
primitives bezeichnet, sind eigenständig und lösen ein gezieltes Problem, wie
zum Beispiel das erstellen eines assymetrischen Schlüsselpaares. Ein Protokoll
kann nur als sicher eingestuft werden, sofern die atomic primitives
ordnungsgemäß genutzt werden und diese für sich selbst als sicher zählen. Als
sicher werden im ersten Kapitel \cite{modern} alle Protolle und atomic
primitives bezeichnet, die theoretisch mit vorhandener Rechenleistungen nicht
errechnet werden können.

