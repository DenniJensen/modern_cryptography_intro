\subsection{Datenaustausch}
\subsubsection{Daten empfangen}
Um Daten im Freenet zu empfangen, wählt oder berechnet der Nutzer einen Hash.
Dieser Hash wird dann als Request-Nachricht über eine Kette von Requests und
mit einer gegebenen hops-to-live-Zahl versendet. Als Start dient der eigene Knoten.
Der Knoten, bei dem die Request-Nachricht ankommt, überprüft, ob der gesuchte
Hash vorhanden ist. Sollte die Datei gefunden werden, sendet der Knoten
die Datei zurück. Sollte der gesuchte Schlüssel nicht vorhanden sein, wird
der nächste Knoten aus der Routingtabelle genommen, um die Request-Nachricht
weiterzuleiten. Dabei wird die hops-to-live-Zahl um eins verringert. Dieser
Vorgang wiederholt sich so lange, bis der hops-to-live-Wert auf null fällt.
Kann ein Knoten einen Request nicht mehr weiterleiten, wird der vorherige
Knoten benachrichtigt. Dieser sucht sich einen neuen Knoten aus der
Routingtabelle und versendet den Request neu. Sollte der hops-to-live-Wert auf
null fallen und es wurde keine Datei mit dem Schlüssel gefunden, wird eine
Fehlernachricht bis zum Anfang der Kette weitergeleitet.
Wird die Datei gefunden, wird dieser Hash in allen Knoten eingetragen, die
Bestandteil der Request-Kette waren.

\subsubsection{Daten verteilen}
Verteilen von Daten ist ähnlich wie das Empfangen von Daten. Der Nutzer
berechnet den Hashwert für die Datei und versendet diese an den eigenen Knoten.
Dieser überpüft, ob der Hashwert schon vorhanden ist. Sollte das der Fall sein,
kommt es zur Kollision und der Nutzer muss einen neuen Schlüssel berechnen.
Sollte der Schlüssel nicht vorhanden sein, leitet der Knoten den Request
weiter, bis der hop-to-live-Wert aufgebraucht ist oder die Datei gefunden
wurde. Sollte die Datei gefunden werden, kommt es wieder zur Kollision und die
Fehlernachricht wird bis zum Initialknoten durchgereicht, der dann wieder einen
neuen Schlüssel berechnen muss. Sollte die Datei nicht gefunden werden, wird
die Datei in allen Knoten eingefügt, die an der Request-Kette beteiligt waren.
Sollte ein Knoten keinen Speicherplatz für die neue Datei verfügbar haben,
wird die am wenigsten aufgerufene Datei ersetzt. Freenet garantiert somit keine
permanente Persistierung von Daten sicher.

\subsubsection{Daten verwalten}
Der Speicher eines Knotens wird als LRU (Last Recent Used) cache verwaltet.
Die Dateien sind nach letzt genutzt sortiert. Kommt beim Knoten eine neue Datei
zum speichern und der Speicher ist voll, wird die am wenigsten genutzte Datei
vertrieben, bis wieder Speicherplatz vorhanden ist. Der Eintrag in der
Routingtablle bleibt vorerst bestehen. Somit kann eine Kopie der Datei
wieder beschafft werden, wenn genug Speicherplatz vorhanden ist. Sollte sich
der Knoten mit weiteren neuen Dateien füllen, verschwindet auch der Eintrag
in der Routingtabelle der entfernten Datei.

Dateien im Freenet sind nicht permanent verfügbar. Sollten alle Knoten
entscheiden die selbe Datei aus ihren Speicher zu vertreiben, ist die im
kompletten Netzwerk nicht mehr verfügbar.

Wird eine veraltete Datei weiterhin genutzt, bleibt sie solange im Speicher,
wie sie auch abgefragt wird. Jede Datei ist im Freenet verschlüsselt und
benötigt einen Schlüssel zum Entschlüsseln. Dieser Schlüssel ist völlig
unabhängig von dem Content-Hash-Key.
Ohne Schlüssel kennt der Betreiber den Inhalt der Datei nicht und ist somit
rechtlich abgeschützt, da jedes Wissen vom Inhalt abgestritten werden kann.

\subsection{Knoten hinzufügen}
Ein neuer Knoten wählt zufälligen Seed und verschickt diesen als Nachricht über
das Freenet-Netzwerk. In dieser Ankündigungsnachricht ist die Adresse und dem
Hash des Seeds des neuen Knoten enthalten.

Bei Einkunft einer Ankündigung eines neuen Knotens, generiert der Empfänger
einen zufälligen Seed, XOR'ed diesen mit dem empfangenem Hash und hasht das
Ergebnis um eine Verbindlichkeit herzustellen. Danach leitet der Empfänger
Knoten den neuen Hash zu einem zufällig gewähltem Knoten aus seiner eigenen
Routingtabelle. Dieser Prozess wiederholt sich bis der hops-to-live Wert von
der Ankündigung des neuen Knoten aufgebraucht ist.
Der letzte Knoten, der die Ankündigungsnachricht empfängt, generiert nur einen
Seed.
Nach Überprüfung der Vereinbarungen in den jeweiligen Knoten, fügt jeder Knoten
in der Kette den neuen Knoten in seine Routingtabelle ein.
